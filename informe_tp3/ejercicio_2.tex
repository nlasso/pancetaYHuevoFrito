\subsubsection{Interrupt Descriptor Table}
A trav\'es de la IDT, definimos donde est\'a el c\'odigo de las interrupciones que manejaremos.\\
La estructura de una entrada en la IDT est\'a definida en idt.h y en idt.c son iniciadas todas las entradas.\\
Por medio de una macro se cargan las primeras 20 interrupciones del procesador, que van desde la divisi\'on por 0 hasta la interrupci\'on SIMD.\\
Luego son completadas todas las entradas restantes de la tabla con entradas de interrupciones inv\'alidas con el prop\'osito de manejar 
de alguna forma todas las interrupciones posibles. Algunas de estas son definidas nuevamente:\\
\begin{itemize}
 \item Interrupci\'on 0x32: Clock.
 \item Interrupci\'on 0x33: Teclado.
 \item Interrupci\'on 0x50: Servicios del sistema (syscalls).
 \item Interrupci\'on 0x66: Handlers de las banderas.
\end{itemize}

En isr.asm se encuentra el c\'odigo donde atendemos estas interrupciones. Saliendo de las 4 interrupciones mencionadas arriba (clock, teclado, syscall, bandera),
todas las interrupciones seran atendidas de una forma similar (para esto usamos un macro). Se realizan las escrituras pertinentes en pantalla y despues se desalojara la 
tarea que la causo. Es importante notar que no todas las interrupciones se imprimen igual, pues algunas traen opcode, asi que en pantalla tenemos un array que nos indica
cuales instrucciones tienen opcode y cuales no.

La estructura de una entrada de la idt, definida en idt.h, es la siguiente:\\
\begin{itemize}
 \item offset\_0\_15: primeros 16 bits del offset al entry point, que atender\'a la interrupci\'on
 \item segsel: selector de segmento de codigo de nivel 0 la gdt
 \item attr: atributos de la entrada: Present, DPL, D. Esto var\'ian seg\'un si la interrupci\'on es de Reloj o Teclado que llevan DPL = 00b
o Servicios o Banderas  cuyo DPL = 11b.
 \item offset\_16\_31: segundos 16 bits del offset al entry point.
\end{itemize}


\begin{tabular}{l l l l l}
Indice & Descripcion & P & DPL & D\\

\hline
0...19 & Ins del procesador & 1 & 0 & 1 \\
32 & Clock & 1 & 0 & 1\\
33 & Teclado	 & 1 & 0 & 1\\
80 & Servicios & 1 & 3 & 1\\
102 & Banderas & 1 & 3 & 1\\
\end{tabular}

\subsubsection{Proceso para activar interrupciones}

Para poder activar todas estas interrupciones y sus respectivos handlers se siguen los siguientes pasos:\\
\begin{itemize}
 \item Mediante el uso de la instrucci\'on LIDT [IDT\_DESC], cargamos el principio del array donde tenemos cargados todas las interrupciones
 \item Por \'ultimo se deshabilita, se resetea y se vuelve a habilitar el pic que obtiene las interrupciones.\footnote{Las funciones de 
deshabilitar, habilitar y resetear fueron provistas por la c\'atedra.}
\end{itemize}
