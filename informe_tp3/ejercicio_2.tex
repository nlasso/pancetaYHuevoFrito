\subsubsection{Interrupt Descriptor Table}
A trav\'es de la IDT, definimos d\'onde est\'a el c\'odigo de las interrupciones que manejaremos.\\
La estructura de una entrada en la IDT est\'a definida en idt.h y en el idt.c cargamos todas las entradas.\\
Por medio de una macro cargamos las primeras 20 interrupciones del procesador, que van desde la divisi\'on por 0 hasta la interrupci\'on SIMD.\\
Luego llenamos el resto de la tabla con entradas de interrupciones inv\'alidas, siendo un total de 256 entradas. Algunas de estas se definir\'an de nuevo como entradas de interrupciones de reloj, teclado, servicios y banderas.\\
En isr.asm se encuentra el c\'odigo donde atendemos estas interrupciones. Por ahora de la 0 a la 19 s\'olo se imprime el c\'odigo de error en pantalla.\\

La estructura de una entrada de la idt, definida en idt.h, es la siguiente:\\
offset\_0\_15: primeros 16 bits del offset al entry point, que atender\'a la interrupci\'on\\
segsel: selector de segmento de codigo de la gdt\\
attr: atributos de la entrada: Present, DPL, D\\
offset\_16\_31: segundos 16 bits del offset al entry point.\\

Todas las interrupciones, excepto las de servicios y banderas, tendr\'an en attr el siguiente valor:\\
0x8E00 = 1000 1110 0000 0000\\
Más detalladamente de izquierda a derecha:\\
1: Present\\
00: DPL = 0\\
0D110 000: D = 1 (Size of gate 32 bits), los demas bits caracterizan el tipo de interrupci\'on que es 'Interrupt'.\\

Las interrupciones de servicios y de banderas, en attr tienen el valor 0xEE00, diferenci\'andose en el DPL que tiene en este caso el valor 3, ya que estas interrupciones ser\'an llamadas por las tareas.\\

\begin{tabular}{l l l l l}
Indice & Descripcion & P & DPL & D\\

\hline
0...19 & Ins del procesador & 1 & 0 & 1 \\
32 & Clock & 1 & 0 & 1\\
33 & Teclado	 & 1 & 0 & 1\\
80 & Servicios & 1 & 3 & 1\\
102 & Banderas & 1 & 3 & 1\\
\end{tabular}




