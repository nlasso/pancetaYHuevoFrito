\documentclass[a4paper,10pt,twoside]{article}

\usepackage[top=1in, bottom=1in, left=1in, right=1in]{geometry}
\usepackage[utf8]{inputenc}
\usepackage[spanish,es-ucroman,es-noquoting]{babel}
\usepackage{setspace}
\usepackage{fancyhdr}
\usepackage{lastpage}
\usepackage{amsmath}
\usepackage{amsfonts}
\usepackage{verbatim}
\usepackage{graphicx}
\usepackage{float}
\usepackage{algorithmic}
\usepackage{tikz}
\usepackage{ gensymb }
\usetikzlibrary{calc}
\usetikzlibrary{decorations.pathreplacing}


% Evita que el documento se estire verticalmente para ocupar
% el espacio vacío en cada página.
\raggedbottom


%%%%%%%%%% Configuración de Fancyhdr - Inicio %%%%%%%%%%
\pagestyle{fancy}
\thispagestyle{fancy}
\lhead{RTP2, Organización del Computador II}
\renewcommand{\footrulewidth}{0.4pt}
\cfoot{\thepage /\pageref{LastPage}}

\fancypagestyle{caratula} {
   \fancyhf{}
   \cfoot{\thepage /\pageref{LastPage}}
   \renewcommand{\headrulewidth}{0pt}
   \renewcommand{\footrulewidth}{0pt}
}
%%%%%%%%%% Configuración de Fancyhdr - Fin %%%%%%%%%%


%%%%%%%%%% Configuración de Algorithmic - Inicio %%%%%%%%%%
% Entorno propio para customizar la presentación del pseudocódigo
\newenvironment{pseudocodigo}
    {\vspace{0.5em} \begin{algorithmic}}
    {\end{algorithmic} \vspace{0.5em}}

% Alinear comentarios a la derecha
\renewcommand{\algorithmiccomment}[1]{\hfill \{#1\}}
%%%%%%%%%% Configuración de Algorithmic - Fin %%%%%%%%%%


%%%%%%%%%% Macros de tikz - Inicio %%%%%%%%%%
% Uso: \registroCuatro{etiqueta}{x}{y}{a4}{a3}{a2}{a1}
\newcommand{\registroCuatro}[7]{
    \ifthenelse{\equal{#1}{}}{}{
        \draw (#2, {#3 + 0.5}) node[anchor=east]{#1};
    }

    \draw   (#2, #3) rectangle +(4, 1) +(2, 0.5) node{#4}
          ++(4, 0)   rectangle +(4, 1) +(2, 0.5) node{#5}
          ++(4, 0)   rectangle +(4, 1) +(2, 0.5) node{#6}
          ++(4, 0)   rectangle +(4, 1) +(2, 0.5) node{#7};          
}

% Uso: \registroOcho{etiqueta}{x}{y}{a8}{a7}{a6}...{a1}
\newcommand{\registroOcho}[9]{
    \def\etiqueta{#1}
    \def\x{#2}
    \def\y{#3}
    \def\aviii{#4}
    \def\avii{#5}
    \def\avi{#6}
    \def\av{#7}
    \def\aiv{#8}
    \def\aiii{#9}
    \registroOchoX    
}
\newcommand{\registroOchoX}[2]{ % Auxiliar - no usar directamente
    \def\aii{#1}
    \def\ai{#2}
    \ifthenelse{\equal{\etiqueta}{}}{}{
        \draw (\x, {\y + 0.5}) node[anchor=east]{\etiqueta};
    }
    \filldraw[fill=white]
        (\x, \y) rectangle +(2, 1) +(1, 0.5) node{\aviii}
        ++(2, 0) rectangle +(2, 1) +(1, 0.5) node{\avii}
        ++(2, 0) rectangle +(2, 1) +(1, 0.5) node{\avi}
        ++(2, 0) rectangle +(2, 1) +(1, 0.5) node{\av}
        ++(2, 0) rectangle +(2, 1) +(1, 0.5) node{\aiv}
        ++(2, 0) rectangle +(2, 1) +(1, 0.5) node{\aiii}
        ++(2, 0) rectangle +(2, 1) +(1, 0.5) node{\aii}
        ++(2, 0) rectangle +(2, 1) +(1, 0.5) node{\ai};
}


% Uso: \registroDieciseis{etiqueta}{x}{y}{a16}{a15}{a14}...{a1}
\newcommand{\registroDieciseis}[9]{
    \def\etiqueta{#1}
    \def\x{#2}
    \def\y{#3}
    \def\axvi{#4}
    \def\axv{#5}
    \def\axiv{#6}
    \def\axiii{#7}
    \def\axii{#8}
    \def\axi{#9}
    \registroDieciseisX
}
\newcommand{\registroDieciseisX}[9]{ % Auxiliar - no usar directamente
    \def\ax{#1}
    \def\aix{#2}
    \def\aviii{#3}
    \def\avii{#4}
    \def\avi{#5}
    \def\av{#6}
    \def\aiv{#7}
    \def\aiii{#8}
    \def\aii{#9}
    \registroDieciseisXX
}
\newcommand{\registroDieciseisXX}[1]{ % Auxiliar - no usar directamente
    \def\ai{#1}
    \ifthenelse{\equal{\etiqueta}{}}{}{
        \draw (\x, {\y + 0.5}) node[anchor=east]{\etiqueta};
    }
    \filldraw[fill=white]
        (\x, \y) rectangle +(1, 1) +(0.5, 0.5) node{\axvi}
        ++(1, 0) rectangle +(1, 1) +(0.5, 0.5) node{\axv}
        ++(1, 0) rectangle +(1, 1) +(0.5, 0.5) node{\axiv}
        ++(1, 0) rectangle +(1, 1) +(0.5, 0.5) node{\axiii}
        ++(1, 0) rectangle +(1, 1) +(0.5, 0.5) node{\axii}
        ++(1, 0) rectangle +(1, 1) +(0.5, 0.5) node{\axi}
        ++(1, 0) rectangle +(1, 1) +(0.5, 0.5) node{\ax}
        ++(1, 0) rectangle +(1, 1) +(0.5, 0.5) node{\aix}
        ++(1, 0) rectangle +(1, 1) +(0.5, 0.5) node{\aviii}
        ++(1, 0) rectangle +(1, 1) +(0.5, 0.5) node{\avii}
        ++(1, 0) rectangle +(1, 1) +(0.5, 0.5) node{\avi}
        ++(1, 0) rectangle +(1, 1) +(0.5, 0.5) node{\av}
        ++(1, 0) rectangle +(1, 1) +(0.5, 0.5) node{\aiv}
        ++(1, 0) rectangle +(1, 1) +(0.5, 0.5) node{\aiii}
        ++(1, 0) rectangle +(1, 1) +(0.5, 0.5) node{\aii}
        ++(1, 0) rectangle +(1, 1) +(0.5, 0.5) node{\ai};
}
%%%%%%%%%% Macros de tikz - Fin %%%%%%%%%%


%%%%%%%%%% Macros misceláneos - Inicio %%%%%%%%%%
\newcommand{\xmm}[1]{\texttt{XMM#1}}
\newcommand{\rax}{\texttt{RAX}}
\newcommand{\rbx}{\texttt{RBX}}
\newcommand{\rcx}{\texttt{RCX}}
\newcommand{\rdx}{\texttt{RDX}}
\newcommand{\rbp}{\texttt{RBP}}
\newcommand{\rsp}{\texttt{RSP}}
\newcommand{\reg}[1]{\texttt{R#1}}
\newcommand{\asm}[1]{\texttt{\uppercase{#1}}}
\newcommand{\INDSTATE}[1][1]{\STATE\hspace{#1\algorithmicindent}}
%%%%%%%%%% Macros misceláneos - Fin %%%%%%%%%%


\begin{document}


%%%%%%%%%%%%%%%%%%%%%%%%%%%%%%%%%%%%%%%%%%%%%%%%%%%%%%%%%%%%%%%%%%%%%%%%%%%%%%%
%% Carátula                                                                  %%
%%%%%%%%%%%%%%%%%%%%%%%%%%%%%%%%%%%%%%%%%%%%%%%%%%%%%%%%%%%%%%%%%%%%%%%%%%%%%%%


\thispagestyle{caratula}

\begin{center}

\includegraphics[height=2cm]{DC.png} 
\hfill
\includegraphics[height=2cm]{UBA.jpg} 

\vspace{2cm}

Departamento de Computación,\\
Facultad de Ciencias Exactas y Naturales,\\
Universidad de Buenos Aires

\vspace{4cm}

\begin{Huge}
Trabajo Pr\'actico Nro 3\\
\end{Huge}
\begin{Huge}
System Pro1gramming - Batalla Bytal
\end{Huge}

\vspace{0.5cm}

\begin{Large}
Organización del Computador II
\end{Large}

\vspace{1cm}

Segundo Cuatrimestre de 2013

\vspace{4cm}

Grupo: \textbf{Frambuesa a la Crema}

\vspace{0.5cm}

\begin{tabular}{|c|c|c|}
\hline
Apellido y Nombre & LU & E-mail\\
\hline
Ignacio, Truffat		& 387/10 & el\_truffa@hotmail.com\\
Lasso, Nicol\'as 			& 763/10 & lasso.nico@gmail.com\\
Rodr\'iguez, Agust\'in	& 120/10 & agustinrodriguez90@hotmail.com\\
\hline
\end{tabular}

\end{center}

\newpage


%%%%%%%%%%%%%%%%%%%%%%%%%%%%%%%%%%%%%%%%%%%%%%%%%%%%%%%%%%%%%%%%%%%%%%%%%%%%%%%
%% Índice                                                                    %%
%%%%%%%%%%%%%%%%%%%%%%%%%%%%%%%%%%%%%%%%%%%%%%%%%%%%%%%%%%%%%%%%%%%%%%%%%%%%%%%


\tableofcontents

\newpage


%%%%%%%%%%%%%%%%%%%%%%%%%%%%%%%%%%%%%%%%%%%%%%%%%%%%%%%%%%%%%%%%%%%%%%%%%%%%%%%
%% Introducción                                                              %%
%%%%%%%%%%%%%%%%%%%%%%%%%%%%%%%%%%%%%%%%%%%%%%%%%%%%%%%%%%%%%%%%%%%%%%%%%%%%%%%


\section{Introducción}
En el siguiente informe se describe el c\'odigo del Trabajo Pr\'actico Nro 3 entregado.
Para la realizaci\'on de este informe, se separó en ejercicios, describiendo los temas de la materia inclu\'idos en el TP: configuraci\'on de la GDT, pasaje a modo protegido, configuraci\'on de la IDT, paginaci\'on, TSS y la organizaci\'on del scheduller.




%%%%%%%%%%%%%%%%%%%%%%%%%%%%%%%%%%%%%%%%%%%%%%%%%%%%%%%%%%%%%%%%%%%%%%%%%%%%%%%
%% Desarrollo                                                                %%
%%%%%%%%%%%%%%%%%%%%%%%%%%%%%%%%%%%%%%%%%%%%%%%%%%%%%%%%%%%%%%%%%%%%%%%%%%%%%%%

\newpage
\section{Desarrollo y Resultados}

\subsection{Ejercicio 1. GDT}

\subsubsection{Global Descriptor Table}
Como ya sabemos, el procesador inicia en ''modo real'', el cual direcciona a 1 MB de memoria y no posee niveles de protecci\'on ni privilegios.\\
Por eso necesitamos que el procesador pase a ''modo protegido'', para direccionar a m\'as memoria y poder manejar distintos niveles de protecci\'on. Nuestro kernel se encargar\'a de hacer esto.\\
\\
Antes de iniciar en modo protegido, es imprescindible tener bien configurado la Tabla de Descriptores Globales, la cual contiene a los descriptores de segmento, con el fin de definir caracter\'isticas de varias \'areas de la memoria.\\
En el enunciado se piden una segmentacion flat, con 4 segmentos que deben direccionar a 1.75 GB: 2 para c\'odigo de nivel 0 y 3 respectivamente, y 2 para datos, de nivel 0 y 3 tambi\'en.\\

La estructura de un descriptor de segmento es la siguiente:\\
\begin{itemize}
  \item L — 64-bit code segment (IA-32e mode only)
  \item AVL — Available for use by system software
  \item BASE — Segment base address
  \item D/B — Default operation size (0 = 16-bit segment; 1 = 32-bit segment)
  \item DPL — Descriptor privilege level
  \item G — Granularity
  \item LIMIT — Segment Limit
  \item P — Segment present
  \item S — Descriptor type (0 = system; 1 = code or data)
  \item TYPE — Segment type
\end{itemize}

Para definir los segmentos que nos requieren, los items importantes son:\\
\begin{itemize}
 \item BASE: este parametro indica el comienzo del segmento. En los 4 casos, este fue 0 ya que se pidió una segmentacion flat.
 \item P: Present, este parametro indica si el segmento est\'a presente en la memoria. El valor en los 4 segmentos es 0x01 ya que efectivamente estaban presentes.
 \item DPL: Nivel de privilegios del descriptor. Dado que se piden dos segmentos de código y dos de datos nivel 0 y nivel3, este 
parametro var\'ia seg\'un cual de estos queremos implementar. Nivel 0 implica DPL = 00b y nivel 3 implica DPL = 11b.
 \item G. Granularity. Este flag indica si el tamaño del descriptor es mayor o menor que 1 Mb. Esto sucede dado que solo se poseen 20 bits para 
indicar el tamaño del segmento. En particular, si G = 1 entonces el valor de los 20 bits ser\'a multiplicado por 4 Kb provocando que con 
 solo 20 bits pueda representar 4Gb de memoria. En nuestro caso queremos un tamaño de 1.75Gb entonces necesitamos G = 1.
 \item Limit: Tamaño del segmento. Va para los 4 segmentos lo mismo.\\
  \indent Tenemos que direccionar a 1.75 GB, que son 1792 MB, que equivalen a 1835008 KB.\\
  \indent Como G vale 0x01, las unidades deben representarse de 4 KB, por eso dividimos por 4.\\
  \indent $\frac{1835008}{4} = 458752$\\
  \indent Pero como la memoria empieza desde el 0, debe ser un n\'umero menos: 458751\\
  \indent $458751 = 0x6FFFF$
 \item Type: Indica si es un segmento de c\'odigo o de datos. Para el segmento de c\'odigo de nivel 0 ponemos el valor de 0x08, indicando 
que es ''Execute only''. para el segmento de c\'odigo de nivel 3 se usa 0x0A, \emph{Read / Execute}. Mientras que para los 2 de datos 
ponemos el valor de 0x02, indicando que son de Read/Write.
\end{itemize}

Tambi\'en se define un segmento que reservado para el \'area de la pantalla en la memoria. Sabemos que empieza en la direcci\'on base
0x000B8000, con un tamaño de 0x0F9F. Dado que se utilizar\'a como un segmemto de datos, su tipo es de Lectura/ Escritura.\\\\
\\
Como se explicar\'a m\'as adelante, tambi\'en necesitamos entradas para cada una de las tareas y sus banderas. Es decir, selectores de TSS. Estos ser\'an definidos de forma din\'amica y no hardcodeados, bas\'andose en la posici\'on de su respectivo TSS.

\subsubsection{Pasaje a modo protegido}

En funci\'on de pasar a ejecutar en modo protegido el manual de \emph{Intel}\footnote{Ver Intel 64 and IA-32 Architectures Software 
Developer's Manual, Volume 3 System Programming Guide} explicita una serie de pasos que se deben seguir para complir con esto.\\
\begin{itemize}
 \item Habilitar A20. al realizar esto habilitamos el acceso a direcciones superiores a 1 Mb de memoria.
 \item Una vez que tenemos configurada la gdt, guardamos su ubicacion en una variable gdt\_desc. Para que luego la instrucci\'on lgdt 
pueda cargar la direccion de comienzo de la GDT.
 \item Seteamos el flag PE del registro CR0, que indica ''Protected Envirnoment''.
 \item Por \'ultimo para pasar a modo protegido hacemos un jmp al comienzo del segmento de c\'odigo de nivel 0.
 \item Una vez ah\'i acomodamos todos los segmentos apuntando a datos de nivel 0 y seteamos la pila del Kernel en 0x27000 seg\'un lo
indicado por el enunciado.
\end{itemize}

\newpage
\subsection{Ejercicio 2 y 5. IDT}

\subsubsection{Interrupt Descriptor Table}
A trav\'es de la IDT, definimos donde est\'a el c\'odigo de las interrupciones que manejaremos.\\
La estructura de una entrada en la IDT est\'a definida en idt.h y en idt.c son iniciadas todas las entradas.\\
Por medio de una macro se cargan las primeras 20 interrupciones del procesador, que van desde la divisi\'on por 0 hasta la interrupci\'on SIMD.\\
Luego son completadas todas las entradas restantes de la tabla con entradas de interrupciones inv\'alidas con el prop\'osito de manejar 
de alguna forma todas las interrupciones posibles. Algunas de estas son definidas nuevamente:\\
\begin{itemize}
 \item Interrupci\'on 0x32: Clock.
 \item Interrupci\'on 0x33: Teclado.
 \item Interrupci\'on 0x50: Servicios del sistema (syscalls).
 \item Interrupci\'on 0x66: Handlers de las banderas.
\end{itemize}

En isr.asm se encuentra el c\'odigo donde atendemos estas interrupciones. Saliendo de las 4 interrupciones mencionadas arriba (clock, teclado, syscall, bandera),
todas las interrupciones seran atendidas de una forma similar (para esto usamos un macro). Se realizan las escrituras pertinentes en pantalla y despues se desalojara la 
tarea que la causo. Es importante notar que no todas las interrupciones se imprimen igual, pues algunas traen opcode, asi que en pantalla tenemos un array que nos indica
cuales instrucciones tienen opcode y cuales no.

La estructura de una entrada de la idt, definida en idt.h, es la siguiente:\\
\begin{itemize}
 \item offset\_0\_15: primeros 16 bits del offset al entry point, que atender\'a la interrupci\'on
 \item segsel: selector de segmento de codigo de nivel 0 la gdt
 \item attr: atributos de la entrada: Present, DPL, D. Esto var\'ian seg\'un si la interrupci\'on es de Reloj o Teclado que llevan DPL = 00b
o Servicios o Banderas  cuyo DPL = 11b.
 \item offset\_16\_31: segundos 16 bits del offset al entry point.
\end{itemize}


\begin{tabular}{l l l l l}
Indice & Descripcion & P & DPL & D\\

\hline
0...19 & Ins del procesador & 1 & 0 & 1 \\
32 & Clock & 1 & 0 & 1\\
33 & Teclado	 & 1 & 0 & 1\\
80 & Servicios & 1 & 3 & 1\\
102 & Banderas & 1 & 3 & 1\\
\end{tabular}

\subsubsection{Proceso para activar interrupciones}

Para poder activar todas estas interrupciones y sus respectivos handlers se siguen los siguientes pasos:\\
\begin{itemize}
 \item Mediante el uso de la instrucci\'on LIDT [IDT\_DESC], cargamos el principio del array donde tenemos cargados todas las interrupciones
 \item Por \'ultimo se deshabilita, se resetea y se vuelve a habilitar el pic que obtiene las interrupciones.\footnote{Las funciones de 
deshabilitar, habilitar y resetear fueron provistas por la c\'atedra.}
\end{itemize}

\newpage
\subsection{Ejercicio 3. Paginaci\'on}

\subsubsection{Kernel, Identity Mapping}
Debemos mapear con Identity mapping las direcciones 0x00000000 a 0x0077FFFF. Para esto fueron necesarios:
\begin{itemize}
 \item 1 Tabla de Directorios de p\'aginas que empieza en la direccion 0x27000.
 \item 2 Entradas de tabla de directorios que abarcan los 1.75 Gb de memoria. 
 \item 2 tablas de p\'aginas. La primer Page table posee sus 1024 entradas completas direccionando desde 0x00000000 hasta 0x003FFFFF 
y tiene como base la direcci\'on 0x28000 y la segunda de 0x40000000 a 0x0077FFFF con direcci\'on base en 0x30000.
\end{itemize}

Las entradas de directorio para Kernel son cargadas de la siguiente manera\footnote{Se pueden considerar a los flags no declarados como
no seteados, es decir, iguales a 0.}:
\begin{itemize}
 \item P = 1.
 \item R/W = 1.
 \item U/S = 0.
 \item Direccion de la Page Table = 0x28000 o 0x30000 seg\'un corresponda la primer o segunda page table.
\end{itemize}

Las entradas de Page Table para el Kernel son cargadas de la siguiente manera\footnotemark[3]:
\begin{itemize}
 \item P = 1.
 \item R/W = 1.
 \item U/S = 0.
 \item Direccion del Page Frame desde 0x00000000 a 0x0077FFFF seg\'un corresponda.
\end{itemize}

A continuaci\'on se detalla un esquema para una mejor comprensi\'on de lo explicado:

\includegraphics[scale=0.6]{imagenes/tablasDePaginasEj3.png}

\subsubsection{Activaci\'on de paginaci\'on}

Luego de armar el directorio de p\'aginas podemos habilitar la paginaci\'on. Para esto seguimo los siguientes pasos:
\begin{itemize}
 \item Cargar en CR3 la direccion al inicio del directorio de p\'aginas.
 \item Setear el bit mas significativo del registro CR0.
\end{itemize}


\newpage
\subsection{Ejercicio 4. Paginaci\'on de tareas}

Para la paginaci\'on de tareas se inicializa un directorio de p\'aginas con 3 entradas por tarea. Las primeras 2 mapean al kernel, y la tercera es para la tarea, la cual tiene 4 entradas de p\'aginas de tabla: 2 para c\'odigo que apuntan al mar, una para la pila de nivel 0 y otra para el ancla.

\includegraphics[scale=0.6]{imagenes/paginacion_tareas.png}

\newpage
\subsection{Ejercicio 6. TSS}

Para que el procesador pueda despachar, ejecutar o suspender m\'ultiples tareas, es necesario salvar el estado de las mismas. La 
arquitectura provee mecanismos para esto. El segmento de estado (TSS, Task State Segment), es el que se encarga de almacenar la 
informaci\'on del estado de una tarea.\\

Una tarea est\'a identificada por el selector de segmento de su TSS. Y a su vez la TSS es un segmento, por lo tanto debe estar descripto 
en la GDT junto con los descriptores de segmento de c\'odigo y datos.\footnote{Ver secci\'on 1 para mas informaci\'on sobre esto.}\\

Tenemos 8 tareas y definimos un total de 18 TSS, uno para cada tarea, uno para cada bandera de tarea, uno para la tarea Idle, y otro lo 
dejamos en blanco para la tarea inicial donde se hace el primer salto.\\
Las entradas de tss idle y la que tss inicial tienen privilegio de kernel, mientras que las dem\'as est\'an configuradas con privilegios 
de usuario.\\

Las TSS se actualizan solas con cada JUMP Far, permiti\'endonos as\'i volver m\'as tarde a esa tarea y no perder la informaci\'on de la 
misma. Por esto mismo es necesario''incializar'' una TSS para que cuando entremos por primera vez la informaci\'on
sea v\'alida. Al momento de inicializar estos segmentos, cada tarea y su flag tendran TSS virtualmente id\'enticas, con la excepci\'on del
 eip y pequeños cambios con respecto a las posiciones de las pilas.\\

Como selectores de segmentos de GDT usamos los que definimos para las tareas (es decir, los de nivel 3), y seteamos el RPL en 0x03 para 
evitar un GPE.\\

Una de las grandes ventajas de estar trabajando con direcciones virtuales es que no tenemos que saber la direcci\'on f\'isica exacta de +
cada tarea para inicializarlas. Sabemos que todas las tareas comparten ciertas direcciones virtuales, asi que seteamos el directorio de 
p\'agina (CR3) correspondiente a esa tarea y podemos usar direcciones id\'enticas para todas las tareas. Como mencionamos antes, las 
banderas recibir\'an datos parecidos, con excepci\'on de las pilas que estar\'an corridas. (El eip que reciban ser\'a indiferente por lo 
que explicamos m\'as abajo).
La pila de nivel 0 ser\'ia un caso especial, pero como es acordado en la secci\'on de paginaci\'on, la mapeamos en la direcci\'on virtual
 0x4000 3000 con el fin de evitar tener que buscarla ahora.

Las TSS de las flags son un caso particular por dos razones. La primera es que el eip no es un valor que sepamos de antemano, sino que 
depende de cada tarea. Al final de cada tarea hay un offset guardado, que sum\'andolo a 0x4000 0000 nos da la direcci\'on virtual 
de la funcion flag. La segunda es que queremos que flag se comporte como una funcion tradicional, es decir, que corra siempre del 
principio hasta el final (o ser interrumpida). En pocas palabras, no nos interesa la posici\'on donde estuvo la \'ultima corrida, sino
que nos interesar\'ia volver siempre al comienzo.\\
Para resolver esto generamos dos funciones que definen el eip del TSS de un flag forma din\'amica, y que deben ser corridas antes de 
saltar a un flag. Por un lado tenemos la funci\'on fetch\_eip, que se encarga de averiguar el offset de la bandera buscando la tarea 
en la memoria (recordar que las tareas "navegan" y en teor\'ia podrian mutar), y la funci\'on reset\_eip, que escribe este dato dentro 
de la TSS.\\

\newpage
\subsection{Ejercicio 7. Scheduller}

El Scheduller es la estructura mas grande y quizas mas compleja de nuestro trabajo. Su funcion es simple, coordinar en 
que orden ocurren los eventos en nuestro OS y determinar ciertas acciones como si una bandera excedio el tiempo que le es dado.\\
\\
Conceptualmente nos imaginamos al Scheduler dividido en dos etapas o dos corridas: una corrida de tareas que es interrumpida por una corrida de banderas.
Hay un timer llamado quantum que dictamina cuantos ciclos le queda a la corrida de tareas hasta que sea interrumpido por la corrida de banderas.
La corrida de banderas se ejecuta y una vez terminada vuelve a la corrida de tareas con el quantum reiniciado.
\\
Corrida tarea:
\\
Task 1 -> Task 2 -> Task  3 ->/corrida bandera/ -> Task 4 -> ...
\\
Corrida bandera:
\\
/venir de corrida tarea/ -> Flag 1 -> Flag 2 -> Flag 3 -> Flag 4 -> Flag 5 -> Flag 6 -> Flag 7 -> Flag 8 -> /volver a corrida tarea/
\\
Ademas, separamos los estados del scheduler (llamado contexto en nuestro codigo) en 5 instancias distintas, EN_IDLE, EN_IDLE_TAREA, EN_IDLE_BANDERA, 
EN_TAREA y EN_BANDERA. Para los fines del trabajo, los dos primeros estados son precindibles, (EN_IDLE solo se usa cuando empieza la maquina o nos quedamos,
quedamos sin tareas / EN_IDLE_TAREA tiene las mismas funciones de EN_TAREA) pero decidimos agregarlos para mantener la coherencia de la estructura. De esta 
manera, hay 3 estados cuyas propiedas nos importan resaltar.
\\
EN_TAREA: indica que se esta/estaba corriendo una tarea. Si vuelve al scheduler, se debera continuar con la corrida de tareas o inicializar la corrida de banderas 
dependiendo del quantum.

EN_IDLE_BANDERA: indica que se esta en un idle despues de haber hecho una bandera y usado la interrupcion 66. Al volver al scheduler esta bandera no debe borrarse,
simplemente debe continuarse con la corrida de banderas. (o de haber terminado, volver a la corrida de tareas)

EN_BANDERA: indica que se esta corriendo una bandera. Si vuelve al scheduler, quiere decir que la bandera no termino de ejecutarse (A.k.a no llamo a la int 66), por lo cual 
debe ser desalojada y luego se debe continuar con la corrida de bandera. (o de tareas de haber terminado)

\\
La interrupcion de clock se encarga de realizar todos los saltos y cambios de tareas, exceptuando el salto a idle (que puede ser hecho en cualquier momento). El scheduller
es la structura que le informa hacia donde ir, siguiendo . De esta forma, mantenemos el codigo facilmente segmentado.

Una excepcion interesante es el caso en el que no querramos saltar a ningun lado sino seguir en la tarea actual. Por ejemplo, si me queda una sola tarea y estoy en la corrida
de tareas aun con quantum me gustaria pertenecer en esa tarea. Para esto el scheduller devuelve el selector de segmento 0, el cual es reconocido por el clock como una
instruccion para volver a la tarea anterior (iret) y no realizar ningun salto. (tratar de saltar a una tarea en uso daria error)

%%%%%%%%%%%%%%%%%%%%%%%%%%%%%%%%%%%%%%%%%%%%%%%%%%%%%%%%%%%%%%%%%%%%%%%%%%%%%%%
%% Conclusión                                                                %%
%%%%%%%%%%%%%%%%%%%%%%%%%%%%%%%%%%%%%%%%%%%%%%%%%%%%%%%%%%%%%%%%%%%%%%%%%%%%%%%



\end{document}