La siguiente secci\'on est\'a dedicada a los \emph{Handlers} o manejadores de las interrupciones. Estos son los c\'odigos que se ejecutan
cuando alguna porci\'on del systema produce un error o llama a una interrupci\'on mejor conocida como syscall o servicios del sistema.\\
Nuestro Kernel cuenta con 4 interrupciones que poseen handlers. Estas fueron mencionadas en el punto 3. Procederemos a explicar cada una de
ellas:
\begin{itemize}
 \item Clock: El clock es una interrupci\'on que se ejecuta cada ciertos ticks de reloj. La misma se encarga de buscar el siguiente selector
de segmento según es especificado en la secci\'on 7. y realizar el salto a dicho selector que puede corresponder a una tarea, una bandera o 
la tarea IDLE.
 \item Teclado: La interrupci\'on de teclado cumple la funci\'on de cambiar el estado de la pantalla entre el mapa si la ''m'' es presionada
y el estado de las banderas y las tareas corriendo si la ''e'' es presionada.
 \item Servicios o Syscalls: Esta interrupci\'on brinda al systema una serie de servicios o funciones a las tareas:
  \begin{itemize}
   \item Fondear: Se accede pasando por par\'ametro el n\'umero 0x923.Esta funci\'on permite a la tarea mover por tierra el ancla permitiendole 
mirar de a una p\'agina por vez en tierra. Esta llama a la funci\'on implementada en C anclar() ubicada en mmu.c.
   \item Cañonear: Se accede pasando por par\'ametro el n\'umero 0x83A, la direccion virtual donde voy a disparar y el buffer de 97 bytes
que funciona como misil. B\'asicamente este servicio nos permite escribir en cualquier lugar del mar un buffer de 97 bytes haciendo que
en caso de que en esa direcci\'on se encontrara una tarea enemiga, sus p\'aginas sean corrompidas. Esta llama a la funci\'on canionear()
implementada en mmu.c
   \item Navegar: bajo el n\'umero 0xAEF, recibe las nuevas direcciones de las primer y segunda p\'aginas de c\'odigo de la tarea. Generando
que mi tarea se pueda mover por el mar sin ser atrapada por una tarea enemiga. Esta syscall llama a navegar() implementada en C en mmu.c.
  \end{itemize}
En este caso hemos añadido un handler de error extra que verifica que no sean llamadas por una bandera haciendo nuestro c\'odigo mas seguro.

 \item Bandera: Esta interrupci\'on se encarga de imprimir la bandera y dar la impresi\'on de movimiento como si una bandera flameara. Cabe 
destacar que solo puede ser llamada por una bandera. Si llega a ser llamada por una tarea la misma debe ser desalojada. Esta llama a Bandera()
implementada en sched.c
\end{itemize}

Cabe destacar que las funciones implementadas en C para las syscalls, navegar, canionear y anclar, se encuentran allí dado que manejan p\'aginas
de memoria haciendo que ubicarlas en mmu.c sea lo m\'as conveniente para aprovechar todas las funciones y estructuras utilizadas. En el 
caso de Bandera() se encuentra en sched.c por un raz\'on similar.
