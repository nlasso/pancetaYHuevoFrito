El Scheduller es la estructura mas grande y quizas mas compleja de nuestro trabajo. Su funcion es simple, coordinar en 
que orden ocurren los eventos en nuestro OS y determinar ciertas acciones como si una bandera excedio el tiempo que le es dado.\\
\\
Conceptualmente nos imaginamos al Scheduler dividido en dos etapas o dos corridas: una corrida de tareas que es interrumpida por una corrida de banderas.
Hay un timer llamado quantum que dictamina cuantos ciclos le queda a la corrida de tareas hasta que sea interrumpido por la corrida de banderas.
La corrida de banderas se ejecuta y una vez terminada vuelve a la corrida de tareas con el quantum reiniciado.
\\
Corrida tarea:
\\
Task 1 -> Task 2 -> Task  3 ->/corrida bandera/ -> Task 4 -> ...
\\
Corrida bandera:
\\
/venir de corrida tarea/ -> Flag 1 -> Flag 2 -> Flag 3 -> Flag 4 -> Flag 5 -> Flag 6 -> Flag 7 -> Flag 8 -> /volver a corrida tarea/
\\
Ademas, separamos los estados del scheduler (llamado contexto en nuestro codigo) en 5 instancias distintas, EN_IDLE, EN_IDLE_TAREA, EN_IDLE_BANDERA, 
EN_TAREA y EN_BANDERA. Para los fines del trabajo, los dos primeros estados son precindibles, (EN_IDLE solo se usa cuando empieza la maquina o nos quedamos,
quedamos sin tareas / EN_IDLE_TAREA tiene las mismas funciones de EN_TAREA) pero decidimos agregarlos para mantener la coherencia de la estructura. De esta 
manera, hay 3 estados cuyas propiedas nos importan resaltar.
\\
EN_TAREA: indica que se esta/estaba corriendo una tarea. Si vuelve al scheduler, se debera continuar con la corrida de tareas o inicializar la corrida de banderas 
dependiendo del quantum.

EN_IDLE_BANDERA: indica que se esta en un idle despues de haber hecho una bandera y usado la interrupcion 66. Al volver al scheduler esta bandera no debe borrarse,
simplemente debe continuarse con la corrida de banderas. (o de haber terminado, volver a la corrida de tareas)

EN_BANDERA: indica que se esta corriendo una bandera. Si vuelve al scheduler, quiere decir que la bandera no termino de ejecutarse (A.k.a no llamo a la int 66), por lo cual 
debe ser desalojada y luego se debe continuar con la corrida de bandera. (o de tareas de haber terminado)

\\
La interrupcion de clock se encarga de realizar todos los saltos y cambios de tareas, exceptuando el salto a idle (que puede ser hecho en cualquier momento). El scheduller
es la structura que le informa hacia donde ir, siguiendo . De esta forma, mantenemos el codigo facilmente segmentado.

Una excepcion interesante es el caso en el que no querramos saltar a ningun lado sino seguir en la tarea actual. Por ejemplo, si me queda una sola tarea y estoy en la corrida
de tareas aun con quantum me gustaria pertenecer en esa tarea. Para esto el scheduller devuelve el selector de segmento 0, el cual es reconocido por el clock como una
instruccion para volver a la tarea anterior (iret) y no realizar ningun salto. (tratar de saltar a una tarea en uso daria error)
