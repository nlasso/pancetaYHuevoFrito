Este filtro consiste en procesar una im\'agen para lograr un efecto miniatura. Consiste en que que los objetos se vean pequeños o como de juguetes. Para esto se "desenfoca" una parte superior y otra inferior de la im\'agen, quedando as\'i s\'olo el foco en la parte del medio, logrando tal efecto.


\subsubsection{Implementación en C}
Para esta implementaci\'on procesamos la imagen por bandas.
Para eso calculamos el límite de las bandas. Luego recorremos mediante dos ciclos la banda del medio y la dejamos igual, no hacemos ninguna transformaci\'on.
Luego dentro de un ciclo que cuenta las iteraciones, hago el procesamiento de "desenfoque" de la banda superior primero y luego la inferior.
Para desenfocar una banda, se recorre mediante dos ciclos que me permiten levantar cada color de cada pixel. Teniendo un componente de color de un pixel, calculo el nuevo valor que debe tener el mismo. Para este c\'alculo se procesa la subimagen que hay alrededor del píxel, haciendo el c\'alculo por color, la multiplicamos por la matriz M, y vamos acumulando ese producto. Para evitar la saturacii\'on, al resultado lo dividimos por 6, que es la suma de las componentes de la matriz M. Con el nuevo valor obtenido, lo pasamos a la im\'agen resultante. Seguimos procesando hasta finalizar la banda.
Para incrementar el efecto, lo que hacemos es, por iteraci\'on, guardar el pixel desenfocado, para que en la siguiente iteración lo procese otra vez.

\subsubsection{Implementación en Assembler}


\subsubsection{Resultados}
