El filtro Decode, obtiene un mensaje a partir de los bytes de la entrada, proces\'andolos primero y luego qued\'andose con los bits relevantes para la 
decodificaci\'on.

\subsubsection{Implementación en C}
En un ciclo principal, se recorren los bytes de la fuente, hasta terminar de recorrer la imagen o alcanzar el tamaño recibido por par\'ametro. Dentro de 
este ciclo hay otro, que recorre de a 4 bytes. Los obtiene primero el c\'odigo de la operación a aplicar, luego los bits que ser\'an procesados y luego 
se realiza el algoritmo correspondiente. En cada iteraci\'on del ciclo interior, en una variable que contiene el byte decodificado, inserta los bits 
procesados en la posici\'on que les corresponde. Al terminar con el cuarto byte, pega ese valor en la salida y avanza los punteros y variables para 
continuar con el siguiente grupo de bytes.

\subsubsection{Implementación en Assembler}
El ciclo del algoritmo obtiene en primer lugar 16 bytes de la fuente. Luego, con una m\'ascara obtengo los bits 2 y 3 de cada byte en un registro. Luego 
con otras m\'ascaras, se obtienen los bytes a los que se les debe sumar 1 en un registro y a los que se debe restar 1 en otro. El pr\'oximo paso, niega 
los bits de los bytes cuyo c\'odigo de opraci\'on es 3. A ese resultado, se le suma 1 al registro usando la m\'ascara obtenida anteriormente, y lo mismo 
se hace para restar 1 usando la otra m\'ascara. Despu\'es de eso, se borran todos los bits que no interesan, dejando solo los bits 0 y 1 de cada byte.
Para juntarlos y armar los bytes decodificados, en diferentes registros, se dejan solo los pares de bits que van en las posiciones 0 y 1, 2 y 3, 4 y 5 , 
6 y 7. Luego, con pshufb y las m\'ascaras correspondientes, llevo esos bits manteniendo sus posiciones, pero a los primeros 3 bytes del registro. Y 
finalmente con la instrucci\'on POR se guardan en un solo registro todos los bits en su posici\'on correspondiente. Finalmente se guarda ese registro en 
la posición del output. Como solo los primeros 3 bytes eran reelevantes para el output, el puntero se avanza solo 3 lugares. En la fuente en cambio, 
aumento 12 lugares la posici\'on. Esto se hace as\'i porque cada pixel ocupa 3 bytes, y para formar un byte de la salida se necesitan 4 bytes. Avanzar de
 a 12 la fuente y de a 3 el destino entonces, es conveniente para avanzar de forma m\'as ordenada y segura.

\subsubsection{Resultados}
