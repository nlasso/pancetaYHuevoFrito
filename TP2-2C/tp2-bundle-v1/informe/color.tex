Este filtro consiste b\'asicamente en dados un color y una distancia pasados como par\'ametros, procesa cada pixel de una im\'agen a color evaluando si el color del mismo se "aleja" m\'as de la distancia del par\'ametro, y si eso pasa, el pixel se transforma a escala de grises, sino lo mantiene igual. Esto logra el efecto de resaltar un color en una im\'agen

\subsubsection{Implementación en C}
Mediante dos ciclos anidados, se va recorriendo la im\'agen por cada componente de color de cada pixel. Por cada pixel se levantan sus 3 colores RGB para calcular la distancia a los 3 colores RGB pasados por par\'ametro. Si el color de la im\'agen supera esa distancia, entonces en el pixel que se est\á procesando quedan sus 3 colores iguales, logrando una escala de grises. Si el color no supera la distancia, debe mantenerse tal cual, logrando as\í ser resaltado.

\subsubsection{Implementación en Assembler}


\subsubsection{Resultados}
