\documentclass[a4paper,10pt,twoside]{article}

\usepackage[top=1in, bottom=1in, left=1in, right=1in]{geometry}
\usepackage[utf8]{inputenc}
\usepackage[spanish,es-ucroman,es-noquoting]{babel}
\usepackage{setspace}
\usepackage{fancyhdr}
\usepackage{lastpage}
\usepackage{amsmath}
\usepackage{amsfonts}
\usepackage{verbatim}
\usepackage{graphicx}
\usepackage{float}
\usepackage{algorithmic}
\usepackage{tikz}
\usepackage{ gensymb }
\usetikzlibrary{calc}
\usetikzlibrary{decorations.pathreplacing}


% Evita que el documento se estire verticalmente para ocupar
% el espacio vacío en cada página.
\raggedbottom


%%%%%%%%%% Configuración de Fancyhdr - Inicio %%%%%%%%%%
\pagestyle{fancy}
\thispagestyle{fancy}
\lhead{RTP2, Organización del Computador II}
\renewcommand{\footrulewidth}{0.4pt}
\cfoot{\thepage /\pageref{LastPage}}

\fancypagestyle{caratula} {
   \fancyhf{}
   \cfoot{\thepage /\pageref{LastPage}}
   \renewcommand{\headrulewidth}{0pt}
   \renewcommand{\footrulewidth}{0pt}
}
%%%%%%%%%% Configuración de Fancyhdr - Fin %%%%%%%%%%


%%%%%%%%%% Configuración de Algorithmic - Inicio %%%%%%%%%%
% Entorno propio para customizar la presentación del pseudocódigo
\newenvironment{pseudocodigo}
    {\vspace{0.5em} \begin{algorithmic}}
    {\end{algorithmic} \vspace{0.5em}}

% Alinear comentarios a la derecha
\renewcommand{\algorithmiccomment}[1]{\hfill \{#1\}}
%%%%%%%%%% Configuración de Algorithmic - Fin %%%%%%%%%%


%%%%%%%%%% Macros de tikz - Inicio %%%%%%%%%%
% Uso: \registroCuatro{etiqueta}{x}{y}{a4}{a3}{a2}{a1}
\newcommand{\registroCuatro}[7]{
    \ifthenelse{\equal{#1}{}}{}{
        \draw (#2, {#3 + 0.5}) node[anchor=east]{#1};
    }

    \draw   (#2, #3) rectangle +(4, 1) +(2, 0.5) node{#4}
          ++(4, 0)   rectangle +(4, 1) +(2, 0.5) node{#5}
          ++(4, 0)   rectangle +(4, 1) +(2, 0.5) node{#6}
          ++(4, 0)   rectangle +(4, 1) +(2, 0.5) node{#7};          
}

% Uso: \registroOcho{etiqueta}{x}{y}{a8}{a7}{a6}...{a1}
\newcommand{\registroOcho}[9]{
    \def\etiqueta{#1}
    \def\x{#2}
    \def\y{#3}
    \def\aviii{#4}
    \def\avii{#5}
    \def\avi{#6}
    \def\av{#7}
    \def\aiv{#8}
    \def\aiii{#9}
    \registroOchoX    
}
\newcommand{\registroOchoX}[2]{ % Auxiliar - no usar directamente
    \def\aii{#1}
    \def\ai{#2}
    \ifthenelse{\equal{\etiqueta}{}}{}{
        \draw (\x, {\y + 0.5}) node[anchor=east]{\etiqueta};
    }
    \filldraw[fill=white]
        (\x, \y) rectangle +(2, 1) +(1, 0.5) node{\aviii}
        ++(2, 0) rectangle +(2, 1) +(1, 0.5) node{\avii}
        ++(2, 0) rectangle +(2, 1) +(1, 0.5) node{\avi}
        ++(2, 0) rectangle +(2, 1) +(1, 0.5) node{\av}
        ++(2, 0) rectangle +(2, 1) +(1, 0.5) node{\aiv}
        ++(2, 0) rectangle +(2, 1) +(1, 0.5) node{\aiii}
        ++(2, 0) rectangle +(2, 1) +(1, 0.5) node{\aii}
        ++(2, 0) rectangle +(2, 1) +(1, 0.5) node{\ai};
}


% Uso: \registroDieciseis{etiqueta}{x}{y}{a16}{a15}{a14}...{a1}
\newcommand{\registroDieciseis}[9]{
    \def\etiqueta{#1}
    \def\x{#2}
    \def\y{#3}
    \def\axvi{#4}
    \def\axv{#5}
    \def\axiv{#6}
    \def\axiii{#7}
    \def\axii{#8}
    \def\axi{#9}
    \registroDieciseisX
}
\newcommand{\registroDieciseisX}[9]{ % Auxiliar - no usar directamente
    \def\ax{#1}
    \def\aix{#2}
    \def\aviii{#3}
    \def\avii{#4}
    \def\avi{#5}
    \def\av{#6}
    \def\aiv{#7}
    \def\aiii{#8}
    \def\aii{#9}
    \registroDieciseisXX
}
\newcommand{\registroDieciseisXX}[1]{ % Auxiliar - no usar directamente
    \def\ai{#1}
    \ifthenelse{\equal{\etiqueta}{}}{}{
        \draw (\x, {\y + 0.5}) node[anchor=east]{\etiqueta};
    }
    \filldraw[fill=white]
        (\x, \y) rectangle +(1, 1) +(0.5, 0.5) node{\axvi}
        ++(1, 0) rectangle +(1, 1) +(0.5, 0.5) node{\axv}
        ++(1, 0) rectangle +(1, 1) +(0.5, 0.5) node{\axiv}
        ++(1, 0) rectangle +(1, 1) +(0.5, 0.5) node{\axiii}
        ++(1, 0) rectangle +(1, 1) +(0.5, 0.5) node{\axii}
        ++(1, 0) rectangle +(1, 1) +(0.5, 0.5) node{\axi}
        ++(1, 0) rectangle +(1, 1) +(0.5, 0.5) node{\ax}
        ++(1, 0) rectangle +(1, 1) +(0.5, 0.5) node{\aix}
        ++(1, 0) rectangle +(1, 1) +(0.5, 0.5) node{\aviii}
        ++(1, 0) rectangle +(1, 1) +(0.5, 0.5) node{\avii}
        ++(1, 0) rectangle +(1, 1) +(0.5, 0.5) node{\avi}
        ++(1, 0) rectangle +(1, 1) +(0.5, 0.5) node{\av}
        ++(1, 0) rectangle +(1, 1) +(0.5, 0.5) node{\aiv}
        ++(1, 0) rectangle +(1, 1) +(0.5, 0.5) node{\aiii}
        ++(1, 0) rectangle +(1, 1) +(0.5, 0.5) node{\aii}
        ++(1, 0) rectangle +(1, 1) +(0.5, 0.5) node{\ai};
}
%%%%%%%%%% Macros de tikz - Fin %%%%%%%%%%


%%%%%%%%%% Macros misceláneos - Inicio %%%%%%%%%%
\newcommand{\xmm}[1]{\texttt{XMM#1}}
\newcommand{\rax}{\texttt{RAX}}
\newcommand{\rbx}{\texttt{RBX}}
\newcommand{\rcx}{\texttt{RCX}}
\newcommand{\rdx}{\texttt{RDX}}
\newcommand{\rbp}{\texttt{RBP}}
\newcommand{\rsp}{\texttt{RSP}}
\newcommand{\reg}[1]{\texttt{R#1}}
\newcommand{\asm}[1]{\texttt{\uppercase{#1}}}
\newcommand{\INDSTATE}[1][1]{\STATE\hspace{#1\algorithmicindent}}
%%%%%%%%%% Macros misceláneos - Fin %%%%%%%%%%


\begin{document}


%%%%%%%%%%%%%%%%%%%%%%%%%%%%%%%%%%%%%%%%%%%%%%%%%%%%%%%%%%%%%%%%%%%%%%%%%%%%%%%
%% Carátula                                                                  %%
%%%%%%%%%%%%%%%%%%%%%%%%%%%%%%%%%%%%%%%%%%%%%%%%%%%%%%%%%%%%%%%%%%%%%%%%%%%%%%%


\thispagestyle{caratula}

\begin{center}

\includegraphics[height=2cm]{DC.png} 
\hfill
\includegraphics[height=2cm]{UBA.jpg} 

\vspace{2cm}

Departamento de Computación,\\
Facultad de Ciencias Exactas y Naturales,\\
Universidad de Buenos Aires

\vspace{4cm}

\begin{Huge}
RTP2
\end{Huge}

\vspace{0.5cm}

\begin{Large}
Organización del Computador II
\end{Large}

\vspace{1cm}

Segundo Cuatrimestre de 2013

\vspace{4cm}

Grupo: \textbf{Frambuesa a la Crema}

\vspace{0.5cm}

\begin{tabular}{|c|c|c|}
\hline
Apellido y Nombre & LU & E-mail\\
\hline
B\'alsamo, Facundo		& 874/10 & facundobalsamo@gmail.com\\
Lasso, Nicol\'as 			& 763/10 & lasso.nico@gmail.com\\
Rodr\'iguez, Agust\'in	& 120/10 & agustinrodriguez90@hotmail.com\\
\hline
\end{tabular}

\end{center}

\newpage


%%%%%%%%%%%%%%%%%%%%%%%%%%%%%%%%%%%%%%%%%%%%%%%%%%%%%%%%%%%%%%%%%%%%%%%%%%%%%%%
%% Índice                                                                    %%
%%%%%%%%%%%%%%%%%%%%%%%%%%%%%%%%%%%%%%%%%%%%%%%%%%%%%%%%%%%%%%%%%%%%%%%%%%%%%%%


\tableofcontents

\newpage


%%%%%%%%%%%%%%%%%%%%%%%%%%%%%%%%%%%%%%%%%%%%%%%%%%%%%%%%%%%%%%%%%%%%%%%%%%%%%%%
%% Introducción                                                              %%
%%%%%%%%%%%%%%%%%%%%%%%%%%%%%%%%%%%%%%%%%%%%%%%%%%%%%%%%%%%%%%%%%%%%%%%%%%%%%%%


\section{Introducción}

El lenguaje C es uno de los más eficientes en cuestión de performance, pero esto no quiere decir que sea óptimo para todos los casos o que no haya campos en los que pueda utilizarse una opción mejor.
Para comprobar esto, experimentamos con el set de instrucciones SIMD de la arquitectura Intel. Vamos a procesar imágenes mediante la aplicación de ciertos filtros y estudiaremos la posible ventaja que puede tener un código en Assembler con respecto a uno en C.
Implementaremos los filtros en ambos lenguajes para luego poder comparar la performance de cada uno y evaluar las ventajas y/o desventajas de cada uno.



%%%%%%%%%%%%%%%%%%%%%%%%%%%%%%%%%%%%%%%%%%%%%%%%%%%%%%%%%%%%%%%%%%%%%%%%%%%%%%%
%% Desarrollo                                                                %%
%%%%%%%%%%%%%%%%%%%%%%%%%%%%%%%%%%%%%%%%%%%%%%%%%%%%%%%%%%%%%%%%%%%%%%%%%%%%%%%


\section{Desarrollo y Resultados}

\subsection{Filtro Color}

Este filtro consiste b\'asicamente en dados un color y una distancia pasados como par\'ametros, procesa cada pixel de una im\'agen a color evaluando si el color del mismo se "aleja" m\'as de la distancia del par\'ametro, y si eso pasa, el pixel se transforma a escala de grises, sino lo mantiene igual. Esto logra el efecto de resaltar un color en una im\'agen

\subsubsection{Implementación en C}
Mediante dos ciclos anidados, se va recorriendo la im\'agen por cada componente de color de cada pixel. Por cada pixel se levantan sus 3 colores RGB para calcular la distancia a los 3 colores RGB pasados por par\'ametro. Si el color de la im\'agen supera esa distancia, entonces en el pixel que se est\á procesando quedan sus 3 colores iguales, logrando una escala de grises. Si el color no supera la distancia, debe mantenerse tal cual, logrando as\í ser resaltado.

\subsubsection{Implementación en Assembler}


\subsubsection{Resultados}


\subsection{Filtro Miniature}

Este filtro consiste en procesar una im\'agen para lograr un efecto miniatura. Consiste en que que los objetos se vean pequeños o como de juguetes. Para esto se "desenfoca" una parte superior y otra inferior de la im\'agen, quedando as\'i s\'olo el foco en la parte del medio, logrando tal efecto.


\subsubsection{Implementación en C}
Para esta implementaci\'on procesamos la imagen por bandas.
Para eso calculamos el límite de las bandas. Luego recorremos mediante dos ciclos la banda del medio y la dejamos igual, no hacemos ninguna transformaci\'on.
Luego dentro de un ciclo que cuenta las iteraciones, hago el procesamiento de "desenfoque" de la banda superior primero y luego la inferior.
Para desenfocar una banda, se recorre mediante dos ciclos que me permiten levantar cada color de cada pixel. Teniendo un componente de color de un pixel, calculo el nuevo valor que debe tener el mismo. Para este c\'alculo se procesa la subimagen que hay alrededor del píxel, haciendo el c\'alculo por color, la multiplicamos por la matriz M, y vamos acumulando ese producto. Para evitar la saturacii\'on, al resultado lo dividimos por 6, que es la suma de las componentes de la matriz M. Con el nuevo valor obtenido, lo pasamos a la im\'agen resultante. Seguimos procesando hasta finalizar la banda.
Para incrementar el efecto, lo que hacemos es, por iteraci\'on, guardar el pixel desenfocado, para que en la siguiente iteración lo procese otra vez.

\subsubsection{Implementación en Assembler}


\subsubsection{Resultados}


\subsection{Decodificaci\'on Esteganogr\'afica}

El filtro Decode, obtiene un mensaje a partir de los bytes de la entrada, proces\'andolos primero y luego qued\'andose con los bits relevantes para la 
decodificaci\'on.

\subsubsection{Implementación en C}
En un ciclo principal, se recorren los bytes de la fuente, hasta terminar de recorrer la imagen o alcanzar el tamaño recibido por par\'ametro. Dentro de 
este ciclo hay otro, que recorre de a 4 bytes. Los obtiene primero el c\'odigo de la operación a aplicar, luego los bits que ser\'an procesados y luego 
se realiza el algoritmo correspondiente. En cada iteraci\'on del ciclo interior, en una variable que contiene el byte decodificado, inserta los bits 
procesados en la posici\'on que les corresponde. Al terminar con el cuarto byte, pega ese valor en la salida y avanza los punteros y variables para 
continuar con el siguiente grupo de bytes.

\subsubsection{Implementación en Assembler}
El ciclo del algoritmo obtiene en primer lugar 16 bytes de la fuente. Luego, con una m\'ascara obtengo los bits 2 y 3 de cada byte en un registro. Luego 
con otras m\'ascaras, se obtienen los bytes a los que se les debe sumar 1 en un registro y a los que se debe restar 1 en otro. El pr\'oximo paso, niega 
los bits de los bytes cuyo c\'odigo de opraci\'on es 3. A ese resultado, se le suma 1 al registro usando la m\'ascara obtenida anteriormente, y lo mismo 
se hace para restar 1 usando la otra m\'ascara. Despu\'es de eso, se borran todos los bits que no interesan, dejando solo los bits 0 y 1 de cada byte.\newline
Para juntarlos y armar los bytes decodificados, en diferentes registros, se dejan solo los pares de bits que van en las posiciones 0 y 1, 2 y 3, 4 y 5 , 
6 y 7. Luego, con pshufb y las m\'ascaras correspondientes, llevo esos bits manteniendo sus posiciones, pero a los primeros 3 bytes del registro. Y 
finalmente con la instrucci\'on POR se guardan en un solo registro todos los bits en su posici\'on correspondiente. Finalmente se guarda ese registro en 
la posición del output. Como solo los primeros 3 bytes eran reelevantes para el output, el puntero se avanza solo 3 lugares. En la fuente en cambio, 
aumento 12 lugares la posici\'on. Esto se hace as\'i porque cada pixel ocupa 3 bytes, y para formar un byte de la salida se necesitan 4 bytes. Avanzar de
 a 12 la fuente y de a 3 el destino entonces, es conveniente para avanzar de forma m\'as ordenada y segura.

\subsubsection{Resultados}




%%%%%%%%%%%%%%%%%%%%%%%%%%%%%%%%%%%%%%%%%%%%%%%%%%%%%%%%%%%%%%%%%%%%%%%%%%%%%%%
%% Conclusión                                                                %%
%%%%%%%%%%%%%%%%%%%%%%%%%%%%%%%%%%%%%%%%%%%%%%%%%%%%%%%%%%%%%%%%%%%%%%%%%%%%%%%


\section{Conclusión}

Las instrucciones SIMD (Single Instruction Multiple Data) proveen al programador de una herramienta más efectiva para realizar el mismo conjunto de operaciones a una gran cantidad de datos.

La aplicación del filtros a imágenes era un ejemplo perfecto para probar su eficiencia.

Analizando los resultados de las implementaciones de los 3 filtros, podemos notar:

\begin{itemize}
\item Las operaciones básicas (padd, psub, pmul, pdiv, shifts, etc.) SIMD tienen un costo similar a sus correspondientes operaciones unitarias, pero generalmente requieren algún tipo de preproceso para poder trabajar con los 16 bytes (pack, unpack, shifts) en una sola iteración, por lo tanto, aunque más eficientes, no lo son en una relación directamente proporcional.
\item En el caso que sí hay una relación directamente proporcional es en el acceso a memoria.
\item Además, el acceso a memoria es, por lejos, la operación más costosa de las que implementamos en cada filtro.
\item Por consecuencia directa del ítem anterior, las llamadas a otras funciones (que a su vez, probablemente contengan variable locales) dentro de una iteración provocan estragos en la efectividad de las implementaciones en C.
\item Para poder aprovechar las instrucciones SIMD es un prerequisito que los datos estén contiguos en memoria. Como descubrimos con el filtro Rotar, los datos dispersos nos obligan a hacer múltiples lecturas a memoria y perder tiempo reordenándolos dentro de los registros antes de poder procesarlos.
\end{itemize}

Concluimos que, definitivamente, las instrucciones SIMD, cuando pueden aprovecharse, demuestran una gran eficiencia. Sin embargo, hay que tener algunas consideraciones:

Aunque las imágenes, video y sonido son los primeros candidatos a ser optimizados por paralelización, no todos los procesos pueden ser efectivos y se requiere un análisis profundo de los datos para ver si vale el esfuerzo.

Además, aunque se pueda lograr una gran optimización, no siempre es lo más importante. Ninguno de los filtros implementados demoró más de 1 segundo en ejecutarse completamente. La optimización seguramente es indispensable en transmisiones de video en vivo, pero baja en importancia si tuviese que ser aplicado una sola vez en una aplicación tipo MS Paint.

Las desventajas que podrían opacar a la optimización son:

El código no es portable, únicamente funciona en procesadores que implementan el set de instrucciones AMD64, requiriendo reescrituras para otras plataformas. Sin embargo el código C debería funcionar perfectamente en IA-32, ARM y cualquier otro procesador que tenga un compilador de lenguaje C.

El código es mucho más largo y difícil de entender (por lo tanto mayor posibilidad de tener bugs) que en un lenguaje de más alto nivel como C. Y en pos de la optimización, se llegan a eliminar funciones (poniéndolas inline), lo que genera código repetido, largo y confuso.




\end{document}