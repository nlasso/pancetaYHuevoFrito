\index{Desarrollo|(}
\section{Desarrollo}

En esta secci\'on se explica detalladamente cada uno de los filtros \'utilizados para procesar las im\'agenes:

\subsection{Filtro de color:}

Este filtro consiste b\'asicamente en dados un color y una distancia o threshold pasados como par\'ametros, procesa cada pixel de una im\'agen a color evaluando si el color del mismo se "aleja" m\'as de la distancia del par\'ametro, y si eso pasa, el pixel se transforma a blanco y negro, sino lo mantiene igual.
\subsubsection{Implementaci\'on en C:}
\subsubsection{Implementaci\'on en asm}
\subsubsection{Resultados}

\subsection{Filtro miniature:}

Este filtro consiste en procesar una im\'agen y hacer que los objetos se vean pequeños o de juguetes. Para esto se "desenfoca" una parte superior y otra inferior de la im\'agen, y as\'i s\'olo queda el foco en la parte del medio, logrando dicho efecto.
\subsubsection{Implementaci\'on en C:}
\subsubsection{Implementaci\'on en asm}
\subsubsection{Resultados}

\subsection{Decodificaci\'on Esteganogr\'afica:}

\subsubsection{Implementaci\'on en C:}
\subsubsection{Implementaci\'on en asm}
\subsubsection{Resultados}
\index{Desarrollo|)}