\index{Introducci\'on Teorica}
\section{Introducci\'on Te\'orica}

En el siguiente trabajo se presenta la implementaci\'on de una herramienta para calcular el costo total de la construcci\'on de un puente Pratt-Truss
a partir de la cantidad de secciones, el alto de cada una y el largo total del mismo. Este trabajo tuvo los siguientes objetivos:

\begin{enumerate}
 \item Implementaci\'on de la factorizaci\'on LU con pivoteo para resolver sistemas de ecuaciones lineales:\newline

    Este m\'etodo consiste en descomponer a la matriz A en dos matrices L triangular inferior y U triangular superior mediante el uso de la eliminaci\'on
Gaussiana con pivoteo. La misma difiere de la eliminaci\'on Gaussiana convencional en que para cada columna donde estoy triangulando, busco el m\'aximo
absoluto y permuto dicha fila con la de la diagonal, dejando una matriz de permutacion P y las matrices LU permutadas. P*A = L*U. En este caso dada la 
particularidad de que nuestra matriz es banda, como veremos mas adelante, tuvimos que tener en cuenta que luego de la factorizaci\'on nuestras bandas 
pasar\'ian de ser p y q a p y p+q.

\item Generar una matriz en base al sistema de ecuaciones planteado por las fuerzas que ejercen sobre la estructura. En el desarrollo se explica de 
que manera se genera esta matriz y como se distribuyen los datos de forma que termine quedando una matriz banda la cual tiene sus ventajas ya que sabemos
que el resto de los elementos fuera de la misma ser\'a 0 y no necesitaremos lugar para almacenalos. \newline

\item Experimentar sobre la mejor combinaci\'on para obtener la fuerza m\'axima mejor distribuida y la menor cantidad de pilares.\newline
\end{enumerate}

\subsection{Demostraci\'on de Matriz banda p, q a Matriz banda p, p+q luego de realizar factorizaci\'on LU con pivoteo}
Dada una matriz banda A $\in \Re^{nxn} / a_{ij} \in A, a \neq 0 \Longleftrightarrow j+i > p \wedge j-i > q$, siendo p la cantidad de columnas antes de la
diagonal y q la cantidad de columnas despu\'es de la diagonal. Quiero ver que luego de aplicar la eliminaci\'on Gaussiana con pivoteo q = p + q.\newline

Entonces se que por cada paso de la eliminaci\'on, la fila por la que voy a poder permutar va a estar entre las filas $i \leq k \leq i + p$. 
Puesto que $\forall k > i + p, a_{kj} = 0$ por su condici\'on de matriz banda.\newline
\newline
\begin{center}
p = 1, q = 2
$\begin{pmatrix}
a_{1,1} & a_{1,2} & a_{1,3} & 0 & 0 & \cdots & 0 \\
a_{2,1} & a_{2,2} & a_{2,3} & a_{2,4}& 0 & \cdots & 0 \\
0 & a_{3,2} & a_{3,3} & a_{3,4} & a_{3,5}& \cdots & 0 \\
\vdots  & \vdots & \vdots  & \ddots & \vdots & \vdots \\
0 & 0 & 0 & 0 & \cdots & a_{n, n-1} & a_{n,n}
\end{pmatrix}$ 
\end{center}
\newpage
Supongo ahora mi peor caso ser\'ia permutar por la fila k = i + p ya que ser\'ia la \'ultima fila y la m\'as a la derecha por la que podr\'ia permutar 
y mover mis bandas lo m\'as posible. Ahora, $\forall 0 < j < n, a_{ij} = a_{kj}$ pero como la fila k estaba comprendida entre $k-p < j < k + q$ y la 
fila i antes de la permutaci\'on esta comprendida entre $i-p < j < i + q$ y adem\'as $k = i + p$, entonces en conclusi\'on la fila i luego de la 
permutaci\'on queda comprendida entre $i < j < i + p + q$ entonces $q = p+q$.
\newline
\begin{center}
$\begin{pmatrix}
a_{1,1} & a_{1,2} & a_{1,3} & 0 & 0 & \cdots & 0 \\
0 & a_{3,2} & a_{3,3} & a_{3,4} & a_{3,5}& \cdots & 0 \\
0 & a_{2,2} & a_{2,3} & a_{2,4}& 0 & \cdots & 0 \\
\vdots  & \vdots & \vdots  & \ddots & \vdots & \vdots \\
0 & 0 & 0 & 0 & \cdots & a_{n, n-1} & a_{n,n}
\end{pmatrix}$
\end{center}

Aqu\'i se nota como luego de la permutaci\'on entre la fila 2 y 3, la nueva fila 2 est\'a shifteada a la derecha una
posici\'on de forma que su primer valor forma parte de la diagonal y todos los elementos de la banda p pasan a estar en la banda q haciendo que se expanda 
hasta p + q en el peor caso.


\index{Introducci\'on Te\'orica|)}
