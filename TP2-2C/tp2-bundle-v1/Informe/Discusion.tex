\index{Discusi\'on|(}
\section{Discusi\'on}

\subsection{ Realizando cambios en los par\'ametros se observa la variaci\'on de la fuerza m\'axima ejercida:}
\begin{itemize}
\item \underline{Variando Span:} Al aumentar dicho valor, la fuerza m\'axima ejercida en la estructura aumenta. Al disminuir se achica dicha fuerza.\newline
Adem\'as se aprecia que el aumeto es escalonado, repitiendose valores de fuerza m\'axima para valores cercanos de diferencia de span. Cabe mencionar que span $>$ n.

\item \underline{Variando n:} Se aprecia en los resultados que en casi todos los casos, al aumentar n, aumentan las fuerzas, mientras que al disminuirlo disminuyen.\newline
Ademas se aprecia un caso particular en el que pasa lo opuesto. \newline

\item \underline{Variando $C_i$:} Se observa que al aumentar los valores de ci, tambien aumentan las fuerzas maximas. En contrapartida se nota la disminucion cuando estos valores disminuyen.

\end{itemize}
\subsection{Realizando cambios en los par\'ametros para observar los cambios en el costo:}

\begin{itemize}
\item \underline{Variando n:} Como ya dijimos al aumentar dicho valor, aumentan las fuerzas y por lo tanto a la hora de analizar el costo esto puede producir perdidas, como se ven en los resultados se insertar\'an m\'as pilares y el costo de dicha estructura ser\'a mucho m\'as elevado.

\item \underline{Variando h:} Al aumentar la altura la fuerza m\'axima disminuye, en algunos casos se logra cambiar el lugar en donde se encuentraba la fuerza m\'axima y ahora tomara el lugar una fuerza que se encuentra en otro lugar la cual mediante este cambio tuvo un crecimiento.\newline
Se observa que en valores muy bajos y muy altos los costos pueden aumentar porque se obtiene una fuerza en m\'odulo mayor y por lo tanto podriamos necesitar la ayuda de pilares en la estructura. El valor mas \'optimo se logra encontrar en un valor ni muy alto ni muy bajo, que es donde la fuerza m\'axima es lo menor posible y nos puede ahorrar pilares costosos.

\item \underline{Variando $F_{MAX}$:} Se observa claramente en los resultados tomados, que al aumentar el $F_{MAX}$ de par\'ametro se consigue una disminuci\'on del costo, ya que se necesitar\'ian menos cantidad de pilares para cubrir. De forma contraria al disminuirlo a muy bajos valores se observa que se deben tomar demasiados pilares e incluso, si $F_{MAX}$ es muy bajo puede que no llegue a cubrir de manera satisfactoria los pilares colocados.

\item \underline{Variando Span:} Como sab\'iamos por lo mencionado en la seccion 4.2 (mas arriba), sabemos que al aumentar span, aumentan las fuerzas y como se pueden ver en los resultados que obtuvimos esto puede ocasionar un costo elevado de la estructura puesto a que m\'as probable que tengo que utilizar una mayor cantidad de pilares que son costosos.

\item \underline{Variando $C_i$:} Se observa en las experimentaciones que al aumentar dicho valor necesitamos a veces mas cantidad de pilares, lo cual nos produce m\'as costo.
En contrapartida al disminuirlo, se observa que la fuerza m\'axima disminuye y por lo tanto necesitamos menos pilares para cubrir.

\end{itemize}

\index{Discusi\'on|)}
