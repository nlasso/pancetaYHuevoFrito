\index{Conclusiones|(}
\section{Conclusiones}

\subsection{Realizando cambio en los parametros se observa la variacion de la fuerza maxima ejercida:}

\begin{itemize}

\item \underline{Variando Span:} Sabemos por resultado que la fuerza m\'axima que se ha encontrado es una fuerza horizontal, sabemos que esta fuerza var\'ia dependiendo del valor span, esto se da porque al aumentar span, aumentan los valores de cosenos y disminuyen los de seno, y como coseno afecta a las fuerzas horizontales, la fuerza m\'axima aumenta. Caso inverso es si span disminuye.

\item \underline{Variando n:} ?`Por qu\'e ocurre que al aumentar n, en general aumentar la fuerza m\'axima?
Esto ocurre por que al aumentar el "n", hay mas cantidad de $C_i$ y entonces todas las dem\'as fuerzas se tensan m\'as y aumentan su fuerza en general.
Conclusi\'on: Si deseamos que la fuerza m\'axima de la estructura no sea muy grande, no es conveniente dividir en gran cantidad de $n$ el span. 

\item \underline{Variando $C_i$:} ?`Por qu\'e cuando aumenta $C_i$ aumenta la fuerza m\'axima y al disminuir $C_i$ disminuye la misma?
Esto se consigue porque si bien la fuerza m\'axima es horizonal y los $C_i$ interfiere en los links vertiles, los $C_i$ llegan tambi\'en a afectar los links horizontales. Esto se da por los links que est\'an colocados de forma hipotenusa, estos a trav\'es del seno y coseno le afectan y afecta a los links horizontales como verticales. Al aumentar el $C_i$, por la f\'ormula Fj*sen(o)+Fi+ci = 0 se observa que Fj y Fi se hacen m\'as chicos e incluso negativos, y si miramos el otro sector donde conecta el Fj, tendremos Fh-Fj*cos(o)-Fk = 0, siendo $Fk$ la fuerza m\'axima, como $Fj$ es negativo y disminuia su valor, -Fj*cos(o) sera positivo y da un valor m\'as grande que antes de recibir el aumento del $Ci$, por consiguiente $Fk$ va a tener que aumentar para que la igualdad se mantenga y por lo tanto aumentara su fuerza m\'axima.\newline
Mismo caso pero en sentido opuesto ocurrira si $C_i$ disminuye.


\end{itemize}

\subsection{Realizando cambio en los parametros para observar los cambios en el costo:}

\begin{itemize}

\item \underline{Variando n:} ?`Por qu\'e el valor n aumenta el costo?
Como ya dijimos al aumentar n se aumentan las fuerzas y por lo tanto se necesitar\'an m\'as pilares para reducirlas, adem\'as al aumentar n tendremos m\'as lugares en donde ubicar pilares. Esto \'ultimo mencionado nos sirve de mucha utilidad porque podremos poner muchos pilares lo que nos permitir\'ia construir una estructura mucho m\'as segura aunque el costo sea muy elevado.

\item \underline{Variando h:} ?`Por qu\'e al aumentar la altura disminuye la fuerza y al disminuirla aumenta?
Esto se da porque es exactamente lo opuesto al aumentar span, aca aumenta seno y disminuye coseno, a su vez esto afecta en la insercion de pilares, y en la fuerza m\'axima que soporta el puente.\newline
?`Conviene tener mucha o poca altura?
No conviene tener demasiada altura ni tampoco poca, ya que al variar la altura hay fuerzas que disminuyen y fuerzas que aumentan por lo tanto hay que tratar de encontrar el mejor equilibrio posible que nos permita tener el mejor ahorro en pilares.

\item \underline{Variando $F_{MAX}$:} ?`Por qu\'e al aumentar o disminuir el par\'ametro $F_{MAX}$, cambian los costos de la estructura?
Si aumentamos el par\'ametro $F_{MAX}$ lo que pasa es que estamos permitiendo que la estructura soporte hacer fuerzas demasiado grande y nos evitaremos construir pilares costoso. La desventaja de esto es que es probable que si la fuerza es muy grande los links se rompan haciendo que la estrictura del puente no sea segura y por lo tanto ocasionar accidentes. Si disminuimos dicho parametro es probable que se inserten mas cantidad de pilares y el puente sera mas costoso, pero tambien mas seguro.

\item \underline{Variando Span:} ?`Por que al aumentar span, aumenta el costo y viceversa?
Esto pasa porque al aumentar span, aumentamos la fuerza m\'axima, y por lo tanto llegar\'a m\'as rapido al rango $F_{MAX}$ y al superarlo se colocaran pilares en la estructura, aumentando el costo del mismo. Mismo caso si disminuimos el span, claro est\'a que el puente ser\'a m\'as corto.
Es importante destacar que si los cambios no son bruscos es probable que no sea necesaria la insercion de un pilar de m\'as.

\item \underline{Variando $C_i$:} ?'Por qu\'e al aumentar los valores de $C_i$ aumenta el costo y al disminuirlo ocurre lo opuesto?
Como sabemos por lo analizado y explicado en el punto 5.1, al aumentar $C_i$ aumentan las fuerzas y por lo tanto es m\'as probable utilizar m\'as cantidad de pilares para poder neutralizarlas. Esto como sabemos aumenta el costo de toda la estructura como se visualiza en los resultados.
\end{itemize}


\index{Conclusiones|)}